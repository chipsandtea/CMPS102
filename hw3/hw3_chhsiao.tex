\documentclass[11pt]{article}
\usepackage{fullpage,amsthm,amsfonts,amssymb,epsfig,amsmath,times,amsthm,enumitem}

\newtheorem{theorem}{Theorem}
\newtheorem{claim}[theorem]{Claim}


\begin{document}

\begin{center}
{\bf\Large CMPS 102 --- Winter Quarter 2017 --  Homework 3}\\
{\bf Christopher Hsiao - chhsiao@ucsc.edu - 1398305}
\end{center}

\section*{Solution to Problem 1 - Tokens}

You're playtesting a new boardgame from one of your friends in the Game Design major, and they want your help to figure out the optimal strategy. The game has $n$ rounds and in each round 1 $\leq i \leq n$ you must acquire $s_i$ tokens.\\
There are two possible ways to acquire tokens in round $i$:

\begin{enumerate}[label=(\Alph*)]
\item Spend $r$ dollars per token, so $r * s_i$ in total, to acquire all tokens of round $i$.
\item Spend $C$ dollars to acquire the tokens of round $i$ and of the next 3 rounds, independently of how many there are. (If you do this in round $i$, then your next choices concern rounds $i + 4$ and greater).
\end{enumerate}

\textbf{Example.} Suppose $r = 1$, $C = 40$, and the sequence of $s_i$ is $11, 9, 9, 12, 12, 12, 12, 9, 9, 11$. Then the cheapest way to acquire all the tokens is to choose action A for the first three rounds, then action B once, and then action A for the final three rounds.


Give a dynamic programming algorithm for finding the cheapest way to acquire all tokens, given the sequence $s_1, s_2, ..., s_n$. You must describe not only how you determine the value of the optimal sequence of actions, but also the sequence of actions itself.

\noindent\rule{17cm}{0.4pt}

\textbf{OPT Identity}

We define $OPT(j)$ to be the minimum cost needed to collect all the tokens from round $1...j$. 
These $OPT(j $values will be stored in an array of size $n$, where the $j^{th}$ entry in this "OPT Array" corresponds to the $j^{th}$ entry in the input array.
\begin{equation}
	OPT(j) = 
	\begin{cases}
	0 & \text{if} \ j = 0\\
	\sum_{i=1}^{j} r \cdot s_i & \text{if} \ j = \{1, 2, 3\}\\
	min\{(r \cdot s_i) + OPT(j - 1), C + OPT(j-4)\} & \text{if} \ j \geq 4
	\end{cases}
\end{equation}

\begin{proof}
\textbf{For $j = 0$:} \\
The minimum cost must be $0$, since there have been $0$ rounds thus far, and we have not had the opportunity to obtain any tokens.\\

\textbf{For $j = \{1, 2, 3\}$:}\\
We must take option $A$ for all choices, since option $B$ requires at least 4 rounds. The first time $B$ can be invoked is when $j = 4$ where option $B$ is invoked on the first index.\\

\textbf{For $j \geq 4$:}\\
Our job is to consider our options between $A$ and $B$, and pick the smaller one. Option $A$ is to calculate the cost of the current round with $(r \cdot s_j)$, and add it to the running sum of minimum costs, represented by $OPT(j-1)$. Option $B$ is to calculate the cost of four rounds $C$, and add it to $OPT(j-4)$, since we obviously want to avoid counting node values we shouldn't think of, and choosing Option $B$ "occupies" four indexes. This is also where the cost $C$ is paid. All that is left is to take these two values and take the smaller of the two, since our goal is end up with the minimum total cost and it wouldn't make sense to pick a higher cost at any point.
\end{proof}

\textbf{Shadow Array: }

We will call the shadow array BIT, and set it to size $n$. Then, we'll set the values like so:

\begin{equation}
	BIT(j) = 
	\begin{cases}
	0 & \text{if} \ \text{Option A was chosen}\\
	1 & \text{if} \ \text{Option B was chosen}\\
	\end{cases}
\end{equation}

So, we have a bitstring BIT. Notice that all values at $j = \{1,2,3\}$ are 0, since those correspond to the first three executions of the algorithm, where choosing $B$ was not yet possible.

Using BIT, we can then determine the sequence of actions taken to yield the minimum total cost to collect all the tokens across $n$ rounds.\\

\textbf{Sequence Algorithm: }
\begin{enumerate}
	\item Set index $k = n$. Initialize a string $ANS$.
	\item If $BIT(k) = 0$, append $A$ to $ANS$ and decrement $k$. 
	\item If $BIT(k) = 1$, append $B$ to $ANS$ and decrease $k$ by $4$.
	\item Repeat this process until $k \leq 0$.
	\item Reverse $ANS$ to get a string of characters (as we recorded them in reverse order), with each character corresponding to both the choices made and the order in which they were made.
\end{enumerate}

\begin{proof}
The reason why step $2$ is true is because it represents choice $A$, which is localized to only one round. Rounds before and after a round where $A$ was chosen are independent of each other. Remember that the goal of this algorithm is to retrace the steps taken that yield the minimum total cost. Step $3$ is true because since we scroll backwards through the array, when we encounter a decision of $B$, then it must have been the last round in the set of $4$. After writing this decision down, we can jump $4$ ahead, since it's just one decision which spanned four entries in $BIT$. Thus, we can accurately record the choices we made in a string.
\end{proof}

The $OPT$ and $BIT$ arrays are filled at the same time as the algorithm executes $n$ times. To find the value of the minimum cost, we wait until $OPT(n)$ is run, and take its value as the minimum cost. Afterwards, we run the Sequence Algorithm detailed above on the $BIT$ array to get the string $ANS$ corresponding to the decisions made and their order.

Observe that this results in two separate passes over arrays of size $n$. Thus, the runtime of this algorithm is $O(n)$.

\pagebreak

\section*{Solution to Problem 2 - Monopoly}

Another Game Design friend is working on a Monopoly spin-off with zoning laws! The board has $n$ spaces, and the value of space $i$ is $P(i)$. However, if you own space $i$ you can't own any of spaces $i - 2, i - 1, i + 1, i + 2$.

\begin{enumerate}[label=(\Alph*)]
\item Give a dynamic programming algorithm for determining the maximum value of the spaces one can own. You do \textbf{not} need to output the corresponding set of spaces.
\item Of course, in real Monopoly, the board wraps around, so let's take that into account as well. That is, assume that if you own space 1, you can not own spaces $n$ or $n - 1$ (besides spaces 2, 3). Solve this problem using DP (ideally, reusing your code for (a) (with tiny modifications) a few times).
\end{enumerate}

\noindent\rule{17cm}{0.4pt}

\subsection*{Part A}
We define $OPT(j)$ to be the maximum value of the first $j$ spaces, and store each of these values in an array called $OPT$ of size $n$. 

\textbf{OPT Identity}

\begin{equation}
	OPT(j) = 
	\begin{cases}
	max_{1, 2, 3} \ P(j) & \text{if} \ j = \{1, 2, 3\}\\
	max\{OPT(j-1), OPT(j-3) + P(j)\} & \ o.w.
	\end{cases}
\end{equation}

\begin{proof}
	\textbf{For $j = \{1, 2, 3\}$:}\\
	For the first three spaces, it is impossible to pick more than one of them at a time, since all of them fall within two units of any other square in either direction. Thus, we must pick the largest of the three values to maximize the total value.
	
	\textbf{For all other values of $j \leq n$:}
	The two options left are whether or not we take the space, or not. If we don't, then $OPT(j) \text{is indistinguishable from} OPT(j-1)$, so we choose that if that yields a greater value. If instead we choose to take the space, then we find the value of the current square $P(j)$, and add it to the only viable space (more than 2 spaces away) $OPT(j-3)$. By calculating these values and comparing them, we select the one which yields the greatest value.
	
	Observe that these are the only options we can make at every $j$, and thus the identity holds.
\end{proof}

To actually calculate the maximum value of the spaces one can own, observe that by definition of $OPT(j)$, we seek the value of $OPT(n)$. Thus, we simply run $OPT(j)$ on all values $1 ... n$, and output the value of $OPT(n)$ value, which can be found in the $n^{th}$ index of our $OPT$ array. We don't need a shadow array for this problem, since the decisions we made to reach this output isn't necessary.

Each execution of $OPT(j)$ runs in $O(1)$ time, and we call $OPT(j) n$ times. Thus, the runtime of this algorithm is $O(n)$.

\subsection*{Part B}

Observe that there can never be \textbf{more than} four adjacent spaces such that none of them are chosen, as the instance where four are chosen mean that this subset of four adjacent spaces are bounded on the left and right by chosen spaces. Thus, it holds that for any five adjacent spaces, at least one of them is in the optimal solution. Let these five spaces be at the wrap-around point (passing "GO" on the monopoly board), with index values $\{n-1, n, 1, 2, 3\}$, of which one is present in the optimal solution. 

We will form five subarrays, of which five spaces are excluded (namely, some space $i$ being taken, and the four excluded squares around $i$). For the set of indexes $\{n-1, n, 1, 2, 3\}$, these subarrays are:

\begin{itemize}
	\item For $i = 1$, SUB$(1) = [4 ... n-2]$
	\item For $i = 2$: SUB$(2) = [5 ... n-1]$
	\item For $i = 3$: SUB$(3) = [6 ... n]$
	\item For $i = n-1$: SUB$(n-1) = [2 ... n-4]$
	\item For $i = n$: SUB$(n) = [3 ... n-3]$
\end{itemize}

We define $A($SUB$)$ to be calling the algorithm defined in part $A$ on the specified sub array SUB. The definition for the best possible value of taking space $i$ is found by calculating $P(i) + A(P(\text{SUB}(i)))$. We will represent this expression for the max value of taking space $i$ as $a_i$. Then, the values for the set $\{n-1, n, 1, 2, 3\}$ is as follows:

\begin{itemize}
	\item $a_1 = P(1) + A(P(\text{SUB}(1)))$
	\item $a_2 = P(2) + A(P(\text{SUB}(2)))$
	\item $a_3 = P(3) + A(P(\text{SUB}(3)))$
	\item $a_{n-1} = P(n-1) + A(P(\text{SUB}(n-1)))$
	\item $a_n = P(n) + A(P(\text{SUB}(n)))$
\end{itemize}

Observe that this applies to all values of $i$, as taking an $i$ excludes the four spaces around it, including wrap-arounds. From earlier, we know that at least one of these indexes are in the optimal solution, so we select the maximum of these five $a_{i-2,\ i-1,\ i,\ i+1,\ i+2}$ values.

We store these values in an $OPT$ array of size $n$, and have each value equal the maximum value up to that index.

This algorithm makes five $O(n)$ operations for each $n$ indexes, we have a runtime of $O(n)$.
\pagebreak

\section*{Solution to Problem 3 - Happy Customers}

You are working on $n$ different projects, but in $m$ hours you are going on vacation. Imagine that for each project $i$, you had a function $f_i$ that told you how happy the people paying for project $i$ would be if out of your $n$ available hours you devoted $0 \leq x \leq m$ to project $i$. Imagine further that each such function has maximum value at most $s_i$, corresponding to the project being fully finished (and thus the clients being perfectly happy).

Give a dynamic programming for determining the most happiness you can generate for your clients by splitting your $m$ hours among the $n$ projects.

\begin{itemize}
\item Because of book-keeping rules, you can't spend fractional hours on a project.
\item The functions $f_i$ are non-decreasing, i.e., for every project $i$, if $t_1 \leq t_2$, then $f_i(t_i) \leq f_i(t_2)$
\item The running time of your algorithm must be $O(n^am^bs^c)$ for some fixed integers $a, b, c \geq 0$.
\end{itemize}


\textbf{OPT Identity: }\\
We define $OPT(i,j)$ to be the maximum happiness possible using the first $i$ projects in the first $j$ hours.

\begin{equation}
	OPT(i,j) = 
	\begin{cases}
	0 & \text{if} \ i = 0\\
	f_i(j) & \text{if} \ i = 1\\
	\underset{k \leq j}{max} \{f_i(k) + OPT(i-1, j-k)\} & \text{if} \ i > 1
	\end{cases}
\end{equation}

\begin{proof}
\textbf{For $i = 0$:}\\
If there aren't any jobs to work, we can't possibly generate any happiness, since no work can be done.

\textbf{For $i = 1$:}
Since there is only one job to work, the maximum amount of happiness we can generate is simply putting in all hours possible into this one task, until you reach $s_i$ or run out of hours.

\textbf{For $i>1$:}
At each new job we have to ask the question "How many hours do we devote to this project?". Let $k$ be the number of hours for the current job $i$. We consider all possibilities of working job $i$ starting with $k=0$, to $k=j$. This represents not working this project at all to putting in all possible hours into this project. For every $k$ value, the amount of happiness is $f_i(k)$ for the current job, plus $OPT(i-1, j-k)$ which represents the maximum happiness considering all jobs from $1 ... i-1$, with $j-k$ hours now available since we dedicate $k$ hours to the current project $i$. The total number of values available at each $OPT(i, j)$ iteration is $j$, since we need to consider $k = \{0...j\}$. Of these $j$ values, we select the one with the maximum happiness, which gets us closer to our goal of finding the maximum amount of happiness across all projects.

Observe that since this represents all possible happiness values and thus all choices we can make, the identity holds.
\end{proof}

\textbf{OPT 2D Array: }

We fill out an $n\ x\ m$ array, to store $OPT(i,j)$ values for all $n$ values of $i$, and all $m$ values of $j$. We will call this array $ARR$, and it follows that $ARR[i][j] = OPT(i, j)$. Observe that we can immediately fill out the entries where $i = 1$ immediately, since $OPT$ refers only to the indexes with one less job, and $OPT(i, j) = 0$ when $i = 0$. With all $i = 1$ entries filled out, we can then fill out all $i = 2$ entries using the $i = 1$ values. We repeat this process until $i = n$. Observe that at each index $i$, we do $j$ work to find all happiness values for every combination of hours possible. Thus, our desired value for the maximum happiness possible is found at $A[n][m]$.

Since $j$ is bounded by $m$, it holds that the amount of work done at each $i$ is $O(m)$ for all $m$ hours. Since there are $n \cdot m$ indexes to do $O(m)$ work at each index, we have a running time of $O(n \cdot m \cdot m) = O(nm^2)$. 

\pagebreak

\section*{Solution to Problem 4 - The Food Bin}

A person is shopping at a store that carries $n$ different food items as packaged goods. A box of the $i$-th item will bring the shopper a benefit of $b_i$ and cost $p_i$ dollars, where $b_i$ and $p_i$ are integers. The shopper would like to receive maximum benefit, subject to the requirement that they can't spend more than $P$ dollars in total. So, the problem is which boxes to take (they can take multiple boxes of the same item. This is known as the integral shopping problem.

In the fractional shopping problem, the same $n$ items are sold in bulk, so the shopper can take as much or as little as they like of each food item, instead of having to buy a whole box or nothing.

\begin{enumerate}
\item Suppose that the items have the same order if we sort them by price from low to high, and if we sort them by benefit from high to low. Give an efficient algorithm to find an optimal solution to the integral shopping problem in this special case, and argue that your algorithm is correct.\\
\textbf{Hint:} Describe the algorithm in English (1 line). Argue correctness in 3 lines.
\item Prove that the fractional version has the following property: the optimal solution can be found by making a series of locally optimal choices. Therefore, give a greedy algorithm to solve this version and prove it's correct.\\
\textbf{Hint:} Ditto
\end{enumerate}
\noindent\rule{17cm}{0.4pt}

\subsection*{Part 1}
\textbf{Algorithm:} Sort the items either by ascending price, or descending benefit (essentially benefit/price ratio, since we know that sorting by ascending price == sorting by descending benefit). Buy as many items starting from index = 1, moving on when you run out of items to buy of that type or if buying one more unit puts you over, until you hit or would go over $P$ (assuming $P$ is greater than the total cost of all items).
\begin{proof}
Let some supposedly better solution be $O$, and let the output of this algorithm to be $S$. Go to the first instant where the solutions differ, and let this index be $i$. Now, since we purchase the item with least price (and by the special case, greatest benefit), that means that at some $i$, $O(i)$ chooses some item with a greater price and/or less benefit. Thus, simply exchange $O(i)$ with $S(i)$, and the solution is guaranteed to at least stay the same, and get better otherwise. This algorithm runs in $O(nlogn)$ to sort the items, then purchasing costs $O(n)$, which gets absorbed into $O(nlogn)$.
\end{proof}

\subsection*{Part 2}
\textbf{Algorithm:} Sort the items based on benefit/price ratio, then purchase items until you spend exactly $P$ dollars, using divisibility to guarantee you can purchase any dollar amount of an item.
\begin{proof}
Given a supposedly more optimal solution $O$ and the algorithms solution $S$, let $i$ be an index where the decisions differ. Since $S(i)$ has the highest benefit/price ratio possible at $i$, that means $O(i)$ chose to purchase some item that was not the highest price ratio it could have purchased at the time. Thus, we can swap $O(i)$ with $S(i)$. If we do this for all indexes where $O$ and $S$ differ, then eventually $O = S$, as we can guarantee that at every point, swapping the choices made won't worsen the solution. Normally when simply choosing by benefit/price ratio we can't guarantee optimality because such an algorithm is too short-sighted, and can't account for combinations that can use the $P$ dollars more efficiently. However, with the power of divisibility, we can guarantee that we always spend $P$ dollars, so we are only concerned with making every dollar count by picking items with the greatest value i.e. best possible benefit/price ratio at every index.
\end{proof}

\end{document}
